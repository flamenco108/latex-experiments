%% TREŚĆ
%\noindent


zażółć gęślą jaźń ZAŻÓŁĆ GĘŚLĄ JAŹŃ

Ala ma Asa, As to Ali pies

My się go boimy, bo szkolony jest

Ola miała kota, lecz zagryzł go As

Wszyscy bardzo się boimy --- by nie zagryzł nas 

Podanie jest pismem urzędowym, którego treścią może być żądanie, wyjaśnienie, odwołanie lub zażalenie w postępowaniu administracyjnym. Można je wnosić w celu wszczęcia postępowania, w czasie jego trwania oraz po wydaniu decyzji administracyjnej.

Podanie można wnieść pisemnie (osobiście lub wysyłając je pocztą). Zgodnie z przepisami można je też wnieść telegraficznie, za pomocą dalekopisu lub telefaksu (te trzy formy mają obecnie niewielkie znaczenie praktyczne). Posłużyć można się także pocztą elektroniczną lub formularzem, umieszczonym na stronie internetowej organu administracji publicznej. W tych dwóch ostatnich przypadkach do podania musi być dołączony bezpieczny podpis elektroniczny. Ostatnią z możliwych form jest wniesienie podania ustnie do protokołu.

Każde podanie musi co najmniej zawierać określenie osoby je wnoszącej (imię i nazwisko w przypadku osoby fizycznej) oraz adres. Musi w nim być również zawarta treść żądania i spełniać inne wymagania, które mogą być określone w przepisach szczególnych. Ponadto podanie wniesione pisemnie lub ustnie do protokołu musi być podpisane przez osobę, która je wniosła. Protokół podpisać musi też pracownik, który go sporządził.

Przepisy prawa nie określają więc form wnoszenia podania - można je wnieść praktycznie w każdej formie, gwarantującej rozpoznanie osoby je wnoszącej. Nie zawierają też szczególnych wymogów co do treści - konieczna jest tylko identyfikacja wnoszącego podanie i jakakolwiek treść.

Cytat:


	Francja, poza podsycaniem nastrojów wojennych Porty (...) i drobną pomocą techniczną nie udzieliła jej większego poparcia, a ze skłóconej i zdemoralizowanej masy rozbitków barskich nie miała Turcja wielkiej pociechy. Ale sukcesy Rosji na Bałkanach nie były na rękę innym konkurentom do schedy po "chorym człowieku Europy" (tak zwano w nowszych czasach upadającą Turcję). Mogła sobie Austria powetować zdobycze rosyjskie aneksjami przy swoich granicach, ale cóż Prusy, które nic by na Turcji nie zyskały, bo z nią nie graniczyły? Akurat wynikły wtedy komplikacje dookoła sprawy polskiej i pojawiły się pierwsze zwiastuny myśli rozbiorczej. Czy więc zamiast rozbierać Turcję nie podzielić się po prosu Polską? Tak więc rozbiory Polski uratowały Portę od gorszego losu. (...)

Traktat w Kucuk Kayanarci oznaczał duże ustępstwa Turcji, lecz byłyby one jeszcze większe, gdyby w tym czasie nie istniała sprawa rozbiorów Polski. (...)

Powiązanie losów Turcji i Polski w tym okresie rozpoczyna nową erę przyjaznych stosunków obu krajów: w tradycji tureckiej i polskiej obraz Turcji przygarniającej rozbitków barskich przesłonił poprzednie obrazy krwawych walk i zmagań, Chocimia i Wiednia.1Reychman, Jan, Historia Turcji, Wrocław, Warszawa, Kraków, Gdańsk, Wyd. Ossolińskich, 1970, str. 174-175

Jednakże nie widać innej korelacji pomiędzy polityką Porty i Polski, która nakazywałaby nam jakąś wdzięczność (raczej odwrotnie): podczas obrad Sejmu Czteroletniego toczyła się kolejna wojna austriacko-rosyjsko-turecka. Groźba utraty kontroli nad umierającą Rzeczpospolitą skłoniła Rosję do szybszego zawarcia pokoju w Jassach i przerzucenia części sił do Polski.

Na początku XVIII wieku, zniknęła Polska jako w pełni samodzielny organizm państwowy, zatem wszelkie kontakty Turcji z naszym krajem mogły odbywać się pośrednio – poprzez stosunki dyplomatyczne z obcymi państwami, lub poprzez kontakty z nie uznanymi oficjalnie organizacjami. I tak też było. W Europie i na Bałkanach ukształtował się nowy polityczny ład, będący ostatecznie efektem Rewolucji Francuskiej - po jednej stronie Austria, Prusy i Rosja, po drugiej tradycyjnie wobec siebie nawzajem serdeczne Francja i - (wróg naszych wrogów) - Turcja, jednak z każdym rokiem słabsza z powodu zacofania, fanatyzmu i korupcji. Dzięki zabiegom posła francuskiego w Stambule, po raz kolejny opowiedziała się po polskiej stronie podczas insurekcji kościuszkowskiej. Gdy w 1794 roku do stolicy dotarło dwóch wysłanników naczelnika powstania, rozpoczęła nawet przygotowania do wojny. I na nich się skończyło. Faktem jest jednak, iż paszowie przygraniczni otrzymali rozkaz pomocy polskim uchodźcom, a po trzecim rozbiorze Imperium Osmańskie ociągało się z uznaniem zaborów. Wynikało to również z prostej kalkulacji opartej na legendach - czy kraj, który do niedawna (w skali imperium) był dla nich tak groźny, może nagle przestać istnieć? Na tej podstawie w Polsce pojawiła się inna legenda: o straży stojącej u wrót nieczynnej polskiej ambasady i stałym miejscu dla polskiego ambasadora na oficjalnych przyjęciach ("Poseł z Lechistanu jeszcze nie przybył!"). Porta chętnie zatrudniała także polskich instruktorów wojskowych, jednak do żadnych zdecydowanych kroków nigdy się nie posunęła.

Okres wojen napoleońskich możemy uznać za niesprzyjający tradycyjnej przyjaźni polsko-tureckiej. Cesarz rozgrywał swoją prywatną politykę dążąc do osłabienia Rosji, zatem przedłużające się jej konflikty z Turcją były mu stanowczo na rękę. Jednocześnie jednak popierał rosyjskie aneksje na Bałkanach, samemu zadowalając się Egiptem i ostrząc sobie zęby na Palestynę i Grecję. W 1809 roku nie wahał się nawet proponować carowi Aleksandrowi wspólnego rozbioru Turcji.

Po Napoleonie zarówno alianci, jak i Francja kontynuowały powolne rozbieranie Imperium Osmańskiego poczynając od jego najdalszych części w Afryce. Próba ponownej aneksji Egiptu przez "silnego sułatana" Muhammada Alego spowodowała utworzenie wspólnej, europejskiej armii interwencyjnej, na której czele miał początkowo w 1831r. stanąć polski generał Wojciech Chrzanowski.

Półwiecze zapoczątkowane upadkiem "Wiosny Ludów", kiedy to znowu polscy uchodźcy większą grupą pod wodzą Józefa Bema uszli do Porty, to nowy nurt w turecko-polskiej polityce emigracyjnej. Już powstanie listopadowe liczyło na turecką pomoc, jednak skończyło się na wyrazach poparcia.

Cytat:


	(...) Husrev Pasza mówił agentom polskim, że "rzecz niebywała, odkąd istnieje wiara muzułmańska, sułtan był pięć razy w wielkim meczecie, aby modlić się za was, psów chrześcijańskich".2Ibidem, str. 223

Stronnictwo Czartoryskiego utrzymywało od lat czterdziestych XIXw. agencję w Stambule domagając się traktowania jej jak przedstawicielstwa dyplomatycznego. Jej program nie zmieniał się niezależnie od temperatury stosunków pomiędzy Portą a Rosją - podtrzymywanie legend o pośle z Lechistanu, przepowiedniach Wernyhory itp. Po naciskach mocarstw rozbiorczych jednak Turcy zmusili ich jednak do opuszczenia kraju, pomimo, że wielu polskich wojskowych w tamtych czasach przeszło na islam (Józef Bem - Amurat Pasza) i ubiegało się o miejscowe obywatelstwo, we wzmacnianiu armii sułtańskiej płonnie upatrując metody na służenie ojczyźnie. 

Dopiero po okresie wojen krymskich Porta ponownie otwiera drzwi dla Polaków. Znowu w jej armii pojawili się polscy oficerowie, rozpoczęto werbunek wszelkich nacji pokrzywdzonych przez Rosję. W tym też czasie do Stambułu przybył Adam Mickiewicz, który tam właśnie zmarł w 1855r. na cholerę. 

Po traktacie paryskim w 1856r. to Polacy raczej zaczęli dostarczać Turcji powodów do wdzięczności. Kilkutysięczna kolonia emigrantów, z reguły - pomiędzy powstaniami - świetnie wykształconych inżynierów, lekarzy i naukowców, oddała temu krajowi nieocenione usługi podczas kontynuowania tzw. reform tanzimatu. Budowali drogi, koleje i miasta, zakładali linie telegraficzne i szpitale na prowincji, kreślili pierwsze plany geodezyjne kraju. Polscy agenci spowodowali nawiązanie stosunków dyplomatycznych Stambułu z Watykanem. Można powiedzieć, że w pewnej mierze Polacy wprowadzali Turcję do Europy.

Powstane Niemiec Bismarcka zakończyło okres idylli w kontaktach polsko-tureckich. Ostatnie podrygi to brak sprzeciwu na sformowanie tzw. "legionu polskiego" w 1877r. na Bałkanach - zdziesiątkowanego w partyzanckich walkach ze wszystkimi. Ten epizod zamknął również nadzieje polskiej emigracji na wygranie sprawy polskiej we współpracy z Turcją. Porta od tej pory już jawnie zaczęła się skłaniać ku spadkobiercy Prus - Rzeszy Niemieckiej.

Dalsze kontakty Turcji i Polski oparte były już tylko o protokół dyplomatyczny, więc opisywanie ich zabrałoby zbyt wiele miejsca w stosunku do ich znaczenia, które w rzeczy samej było nieduże. Turcja realizowała program europeizacji, w który wtłoczył ją Kemal Ataturk. W czasie II Wojny Światowej prowadziła politykę neutralności i nieuznawania faktów dokonanych drogą agresji, zatem nie zamknęła polskiego przedstawicielstwa dyplomatycznego. Jednak antyradzieckie nastroje wśród polityków rządu londyńskiego w krótkim czasie doprowadziły do izolacji polskiej dyplomacji od tureckiej, która w tym czasie bardzieje obawiała się o swoje granice na Kaukazie i Bałkanach. Zagrożenie ze strony komunistycznej Rosji pchnęło ten kraj w ramiona NATO, a w efekcie tego także do kandydowania do EWG.


%\vskip 1.0cm
%\vspace{\stretch{1}}

%\hspace{\fill} Z~poważaniem \hspace{3.0cm}