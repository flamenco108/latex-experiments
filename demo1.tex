zażółć gęślą jaźń ZAŻÓŁĆ GĘŚLĄ JAŹŃ

Ala ma Asa, As to Ali pies

My się go boimy, bo szkolony jest

Ola miała kota, lecz zagryzł go As

Wszyscy bardzo się boimy --- by nie zagryzł nas \footnote{Znana piosenka
  z lat 80-tych.}

Podanie jest pismem urzędowym , którego treścią może być żądanie,
wyjaśnienie, odwołanie lub zażalenie w postępowaniu administracyjnym.
Można je wnosić w celu wszczęcia postępowania, w czasie jego trwania
oraz po wydaniu decyzji administracyjnej.

\section{Co nam wiadomo o podaniu - rozdział}

Podanie można wnieść pisemnie (osobiście lub wysyłając je pocztą).
Zgodnie z przepisami można je też wnieść telegraficznie, za pomocą
dalekopisu lub telefaksu (te trzy formy mają obecnie niewielkie
znaczenie praktyczne). Posłużyć można się także pocztą elektroniczną lub
formularzem, umieszczonym na stronie internetowej organu administracji
publicznej. W tych dwóch ostatnich przypadkach do podania musi być
dołączony bezpieczny podpis elektroniczny. Ostatnią z możliwych form
jest wniesienie podania ustnie do protokołu.

Podanie można wnieść pisemnie \footnote{osobiście lub wysyłając je
  pocztą.

  ale można też emailem.}.

Zgodnie z przepisami można je też wnieść telegraficznie, za pomocą
dalekopisu lub telefaksu (te trzy formy mają obecnie niewielkie
znaczenie praktyczne). Posłużyć można się także pocztą elektroniczną lub
formularzem, umieszczonym na stronie internetowej organu administracji
publicznej. W tych dwóch ostatnich przypadkach do podania musi być
dołączony bezpieczny podpis elektroniczny. Ostatnią z możliwych form
jest wniesienie podania ustnie do protokołu.

\subsection{Cechy szczególne podania właściwego - sekcja czyli
podrozdział}

Każde podanie musi co najmniej zawierać określenie osoby je wnoszącej
(imię i nazwisko w przypadku osoby fizycznej) oraz adres. Musi w nim być
również zawarta treść żądania i spełniać inne wymagania, które mogą być
określone w przepisach szczególnych. Ponadto podanie wniesione pisemnie
lub ustnie do protokołu musi być podpisane przez osobę, która je
wniosła. Protokół podpisać musi też pracownik, który go sporządził.

Każde podanie musi co najmniej zawierać określenie osoby je wnoszącej
(imię i nazwisko w przypadku osoby fizycznej) oraz adres. Musi w nim być
również zawarta treść żądania i spełniać inne wymagania, które mogą być
określone w przepisach szczególnych. Ponadto podanie wniesione pisemnie
lub ustnie do protokołu musi być podpisane przez osobę, która je
wniosła. Protokół podpisać musi też pracownik, który go sporządził.

\subsubsection{Czy każde, czy nie każde - podsekcja czyli
podpodrozdział}

Przepisy prawa nie określają więc form wnoszenia podania - można je
wnieść praktycznie w każdej formie, gwarantującej rozpoznanie osoby je
wnoszącej. Nie zawierają też szczególnych wymogów co do treści -
konieczna jest tylko identyfikacja wnoszącego podanie i jakakolwiek
treść.

Przepisy prawa nie określają więc form wnoszenia podania - można je
wnieść praktycznie w każdej formie, gwarantującej rozpoznanie osoby je
wnoszącej. Nie zawierają też szczególnych wymogów co do treści -
konieczna jest tylko identyfikacja wnoszącego podanie i jakakolwiek
treść.

\paragraph{Co widać - podpodsekcja czyli podpodpodrozdział}

Jednakże nie widać innej korelacji pomiędzy polityką Porty i Polski,
która nakazywałaby nam jakąś wdzięczność (raczej odwrotnie): podczas
obrad Sejmu Czteroletniego toczyła się kolejna wojna
austriacko-rosyjsko-turecka. Groźba utraty kontroli nad umierającą
Rzeczpospolitą skłoniła Rosję do szybszego zawarcia pokoju w Jassach i
przerzucenia części sił do Polski.

Jednakże nie widać innej korelacji pomiędzy polityką Porty i Polski,
która nakazywałaby nam jakąś wdzięczność (raczej odwrotnie): podczas
obrad Sejmu Czteroletniego toczyła się kolejna wojna
austriacko-rosyjsko-turecka. Groźba utraty kontroli nad umierającą
Rzeczpospolitą skłoniła Rosję do szybszego zawarcia pokoju w Jassach i
przerzucenia części sił do Polski.

\subparagraph{Półwiecze - wyróżniony akapit}

Półwiecze zapoczątkowane upadkiem "Wiosny Ludów", kiedy to znowu polscy
uchodźcy większą grupą pod wodzą Józefa Bema uszli do Porty, to nowy
nurt w turecko-polskiej polityce emigracyjnej. Już powstanie listopadowe
liczyło na turecką pomoc, jednak skończyło się na wyrazach poparcia.

Półwiecze zapoczątkowane upadkiem "Wiosny Ludów", kiedy to znowu polscy
uchodźcy większą grupą pod wodzą Józefa Bema uszli do Porty, to nowy
nurt w turecko-polskiej polityce emigracyjnej. Już powstanie listopadowe
liczyło na turecką pomoc, jednak skończyło się na wyrazach poparcia.

Wieczność - wyróżniony podakapit

Upłynęła szczęśliwie, ani się obejrzeliśmy.

Upłynęła szczęśliwie, ani się obejrzeliśmy.


\section*{Cytaty}

Cytat:

Francja, poza podsycaniem nastrojów wojennych Porty (...) i drobną
pomocą techniczną nie udzieliła jej większego poparcia, a ze skłóconej i
zdemoralizowanej masy rozbitków barskich nie miała Turcja wielkiej
pociechy. Ale sukcesy Rosji na Bałkanach nie były na rękę innym
konkurentom do schedy po "chorym człowieku Europy" (tak zwano w nowszych
czasach upadającą Turcję). Mogła sobie Austria powetować zdobycze
rosyjskie aneksjami przy swoich granicach, ale cóż Prusy, które nic by
na Turcji nie zyskały, bo z nią nie graniczyły? Akurat wynikły wtedy
komplikacje dookoła sprawy polskiej i pojawiły się pierwsze zwiastuny
myśli rozbiorczej. Czy więc zamiast rozbierać Turcję nie podzielić się
po prosu Polską? Tak więc rozbiory Polski uratowały Portę od gorszego
losu. (...)

Traktat w Kucuk Kayanarci oznaczał duże ustępstwa Turcji, lecz byłyby
one jeszcze większe, gdyby w tym czasie nie istniała sprawa rozbiorów
Polski. (...)

Powiązanie losów Turcji i Polski w tym okresie rozpoczyna nową erę
przyjaznych stosunków obu krajów: w tradycji tureckiej i polskiej obraz
Turcji przygarniającej rozbitków barskich przesłonił poprzednie obrazy
krwawych walk i zmagań, Chocimia i Wiednia.1Reychman, Jan, Historia
Turcji, Wrocław, Warszawa, Kraków, Gdańsk, Wyd. Ossolińskich, 1970, str.
174-175
